\section{风险与对策}
纵览历年数据,在产能过剩的时代背景下,抽油烟机行业销量稳步上涨,高端产品销量节节攀升,在众多家电产品中一枝独秀,意味着巨大的市场潜力;横比各家企业,抽油烟机行业品牌林立、企业众多,尽管其市场占额日趋稳定,却仍未形成垄断性企业,预示着新兴企业发展的机会.正如十九大报告中指出,“我国社会主要矛盾已经转化为人民日益增长的美好生活需要和不平衡不充分的发展之间的矛盾”,人们对于生活品质与健康环保的需求日益增加;国家也先后出台多项规定,确定了抽油烟机质量标准,将其纳入新能效标识管理体系,并将抽油烟机划入家电下乡项目,对抽油烟机行业及市场做了进一步的规范化调整,为行业发展指明方向.

尽管行业形势总体看来较为适合新兴企业发展,两级抽油烟机也迎合了当下消费者对于抽油烟机性能的消费需求,然而想要在激烈的家电市场竞争中打造出自己的特有品牌,以至逐渐稳定占据足够的市场份额,所需面对的风险仍然十分巨大,所以必须根据可能出现的风险种类提前制定相应对策.

\subsection{市场风险}
\subsubsection{市场风险分析}
\begin{enumerate}
\item 市场需求量

      由于实际的市场需求难以确定或者市场预测的失误,当推出所生产的新产品后,使得产品设计能力超过了市场的实际需求而增加公司的投资风险.

      当下抽油烟机市场中近吸式与大风量设计普遍受宠,但随着消费能力的不断提升与消费要求的不断提高,在解决抽油烟量的基本问题后,绿色健康、环保美观必将成为抽油烟机行业发展的下一个阶段.在行业即将迈入新发展阶段的时期,设计理念不应过多超越当前需求,且市场对于新型产品的需求量难以估计.如果不能较为准确地判断市场需求,两级抽油烟机将面临难以成功打入主流市场的风险.

\item 市场竞争力

      公司生产的产品常常面临着激烈的市场竞争,这种竞争不仅有现有企业之间的竞争,同时还有潜在进入者的威胁.引射器原理最早应用于航空航天领域,空气倍增技术被广泛商用则是戴森出品的高端无扇吹风机.而在抽油烟机领域已有国内大厂商在申请与这二者相关的衍生出的技术专利.鉴于该设计的创新性与突破性,不了解有多少企业计划或正在进入这个市场,对市场信息与竞争烈度的掌握程度不够,预测估计方面或存在较大偏差,使得我们不能够适时提出恰当的、具有针对性的策略.

\item 价格风险

      随着潜在进入者与行业内现有竞争对手两种竞争竞争力量的加剧,各公司会采取“价格战”策略打击竞争对手,因而影响产品价格波动进而影响公司收益.当前,国内抽油烟机行业逐步迈向品牌化,形成以方太、老板等企业为代表的“两超”多强局面.考虑到线下的品牌集中度比线上高出不少,几大吸油烟机巨头在线上的话语权仍低于线下.在吸油烟机新能效标准出台后,一线龙头厂商的线上市场份额越做越大,具备一定技术研发能力的二线品牌对主流品牌的挤压还有一定的承受力,但不具备规模优势的小品牌恐怕面临出局的危险.

\item 市场接受时间

      产品推出后,顾客由于不能及时了解其性能,对我们的新产品持观望、怀疑态度,甚至做出错误的判断从而影响产品的价格.无扇抽风的理念、小巧玲珑的体型、个性设计的外观、两级开放的全新观感固然吸引人,但也使得消费者对于全新产品有着如功率方面与质量方面的天然忧虑,延缓我们产品的市场接受时间,拖长回报周期,并吸引大企业的关注,对于企业在行业内的立足具有较大的威胁.

\end{enumerate}
\subsubsection{市场风险对策}
\begin{enumerate}
\item 从产品角度,企业要降低市场风险,实现长期利益最大,就必须通过不断淘汰旧产品、开发新产品达到最优产品组合.发展电子商务、实现网络化营销;拓展销售渠道,推进集团化管理;发展加工中心,强化生产、质量管理.

\item 从市场客户角度,降低其市场风险,重在开发多样化的客户群,以充分实现其产品价值,降低产品销售的市场风险. 根据市场变化调整产品的结构,选择公司最有利的目标客户,利用强有力的营销组合和客户服务建立高知名度、高美誉度、高忠诚度的品牌形象.

\item 从流程角度,搭建统一的业务应用平台,实现采购、销售、仓储、配送、技术开发、质量、计量集成管理和数据共享,帮助企业科学制定销售、采购、加工和配送计划,提高整个供应链系统的可观性和可控性.规范内部管理,固化运作流程,实现对经营流程各环节的优化和控制,提高企业管控水平,降低经营风险.

\item 从市场信息角度,将招纳专门的信息收集、分析人员,设立一个专门的信息处理部门,及时反馈并处理市场信息,适时适当地应对市场变化.
\end{enumerate}

\subsection{管理风险}
\subsubsection{管理风险分析}
产品的研发投入与收益回报均具有一定周期,设计环节也十分复杂,期间若出现不可抗或不可预测的意外事件或事故,以及宏观经济形势发生较大变化,公司组织结构、管理方法可能不适应不断变化的内外环境,自主研发团队所开发的产品和服务不能跟上消费者需求变化的脚步,且公司刚刚成立,在人员分配、存货管理、激励机制、产品研发等方面会有一些漏洞,都将会对整体运作流程造成影响,大大影响产品的研发进展与公司的回报收益.

\subsubsection{管理风险对策}
\begin{enumerate}
\item 设立规范化、流程化的制度章程.制定合理人员分配方案,设置区域性的组织结构,制定职责明确,奖惩分明,配备完善的制度.加强对管理人员组织结构、管理制度、管理方法等方面的内部培训与外部培训,提高整体素质和经营管理水平.

\item 正确掌握市场的动态,把握市场前进方向,并预先储备相关人才.吸收具有丰富投资管理与运营管理方面经验的专业人才进入公司管理层,吸收具有开发经验与专业背景的人才进入公司核心科研团队,在提高专业人才薪酬和奖励的基础上,为其提供良好的工作与科研条件,并将核心人员吸收为公司股东,实现利益共享,保证核心团队的新陈代谢,为公司发展提供强大动力.

\item 倡导组织创新、思维创新,以适应不断变化的外部环境.
\end{enumerate}


\subsection{财务风险}
\subsubsection{财务风险分析}
财务风险是指企业由于不同的资本结构而对企业投资者的收益产生的不确定影响.财务风险来源于企业资金利润率和接入资金利息率差额上的不确定因素以及借入资金与自有资金的比例大小.借入资金比例越大,风险程度越大,反之则越小.
由于公司初期规模较小,且面向的消费群体有限,财务风险主要体现为以下三个方面:
\begin{enumerate}
\item 资金短缺,不能满足公司快速发展的需要.
\item 财务人员的业务素质较低,会直接导致业务纰漏出现;
\item 资金回收策略不当以及在筹资方面资金结构不合理.
\end{enumerate}

\subsubsection{财务风险对策}
\begin{enumerate}
\item 实行严格的资金借贷和运用审批制度,根据公司发展情况和资金市场成本变化调整资本结构.
\item 加强对业务收入、业务支出、日常现金等的管理,在保持较高流动性的基础上,减少资金占比.为公司扩大投资提供现金流.
\item 使投资项目尽快产生效益,提高资产盈利能力,降低投资风险.
\item 加强对资金运行情况的监控,实施财务预决算制度,最大限度地提高资金使用效率.
\item 建立相应的风险预警机制,加强内部管理,规范章程制度,将可能发生的损失降低到最低程度.
\item 在财务预算中拨留专款购买保险,以规避或应对因意外或其他各种不可抗因素引起的损失与风险.
\item 对财务人员加强技能的培训,提高业务素质和学习适应能力.
\end{enumerate}


\subsection{政策风险}
\subsubsection{政策风险分析}
经济政策风险是指在企业经营期间内,由于所处的经济环境和经济条件的变化,致使实际的经济效益与预期的经济效益相背离.对经济环境和经济条件,应以宏观和微观两个角度进行考察.宏观经济环境与经济条件的变化,是指国家经济制度的变革、经济法规和经济政策的修改、产业政策的调整及经济发展速度的波动.

在产能过剩与厨电行业略显疲软的年代,抽油烟机行业一枝独秀的高速发展得益于我国房地产领域的不断发展、人民改善生活水平的硬性需求以及国家家电下乡政策的引领指导,而这三点在短期内难以改变,故主要考虑的政策风险来源于产业政策调整.

\subsubsection{政策风险对策}
\begin{enumerate}
\item 在国家各项经济政策和产业政策的指导下,汇聚各方信息,提炼最佳方案,统一指挥调度,合理确定公司发展目标和战略.
\item 加强内部管理,提高服务管理水平,降低营运成本,努力提高经济效率,形成公司的独特优势,增强抵御政策风险的能力.
\end{enumerate}


\subsection{技术风险}
\subsubsection{技术风险分析}
由于目前抽油烟机市场上还没有相似技术的产品,所以在初始阶段技术风险相对较小.但由于现代知识更新的加速和科技发展的速度日新月异,致使新技术的生命周期缩短,以及竞争企业的技术不断提高,会对现有产品形成冲击.同时,如果产品开发成功后,不能成功进行大批量生产,仍不能完成风险投资的全过程.另外,随着我国抽油烟机生产技术发展迅猛,业内的人才竞争日益激烈,存在核心技术人员流失的风险.

\subsubsection{技术风险对策}
\begin{enumerate}
\item 拓宽人才招聘渠道,主要以网络招聘、招聘会、内部推荐、报纸刊物为主.可以尝试依托西安交通大学为平台,吸引具有发展潜力的本科生、研究生等作为人才预备.
\item 提高员工职业技能水平和管理水平.
\item 向广大客户征求意见,对产品不断改进、增强竞争力.
\item 引进专业技术人员并不断组建团队,对市场可能的发展方向进行人才预备,保持团队人员的稳定性,并以分配股权的方式共同发展.
\end{enumerate}