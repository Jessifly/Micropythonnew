\section{可行性分析}
\subsection{政策分析}
\begin{enumerate}
\item 2010年3月29日,财政部和商务部印发了《新增家电下乡品种实施方案》将吸油烟机纳入补贴范围内,并明确了新增品种最高补贴限额,抽油烟机最高补贴限额为338元.该方案的实施进一步扩大了吸油烟机行业的市场需求.
\item 2009年《家用电器安全使用年限细则》颁布,规定吸油烟机使用寿命约为8年,促进了吸油烟机的更新换代,同时也带动了吸油烟机的销售.
\item 2016年北京版节能家电鼓舞政策落地,清晰了将对电视机、电冰箱、洗衣机、空调、热水器、吸油烟机、空气净化器、坐便器和自行车等9 类商品进行节能减排财政补助.其间,电视机、电冰箱、洗衣机、空调、热水器、吸油烟机的能效标识须为中国能效标识一级或二级规范.最高补助额度每台800元.
\item 2011年工信部推出《关于加快我国家用电器行业自主品牌建设的指导意见》,意见指出政府会大力支持自主品牌的国际化战略,同时加强对家电区域品牌建设的指导和支持.为家电企业提供了良好的发展环境.
\end{enumerate}

\subsection{财务分析}
\subsubsection{融资与资金使用计划}
我公司处于早期创业阶段,但具有强劲的发展潜力,适当融资对企业成长十分重要.

为保证我创业团队的稳定发展,我们将进行种子轮融资300万元.其中投资方投入资金100万元,我方专利技术抵押借款100万元,公司自筹与众筹合算100万元.

在技术核心的基础之上,以低成本高利润打入市场,初期定位在中高端市场,一方面以较高端产品低的价格快速扩充市场,一方面以高质量赢得用户.在销售策略上,除去传统的代理商销售模式之外,加入线上用户互动平台和线下风力体验馆,以用户自身的体验消除对于新型产品的不信赖感,以互动的方式拉近与顾客的距离和达到无形促销的目的.在融资计划上,初期进行种子轮融资;服务覆盖沙坡村及其附近区域之后,进行天使轮融资,目标为碑林区市场.在碑林区内成功复制商业模式后,将进行A轮融资,进行西安市全市场覆盖,此时公司进入中期阶段.

\begin{longtable}{l|l}
\hline
融资方式  &  覆盖范围  \\
\hline
种子轮融资  &  西安市碑林区沙坡村附近  \\
天使轮融资  &  西安市碑林区内  \\
A轮融资  &  西安市地毯式铺开   \\
B轮融资  &  陕西省铺开至县区  \\
C轮融资  &  全国范围内  \\
\hline
\end{longtable}

公司未来花销如下表显示:

\begin{table}[htbp]
  \centering
  \caption{基础设施花销细节}
    \begin{tabular}{|rrrrr|}
    \toprule
    项目 & 资金(单位:万元) & 数量 & 总计 & 来源 \\
    \hline
    可支配资金 & 200   &       &       &  \\
    厂房    & 20/年  & 1     & 20    & 西安高新区国家级科技企业加速器 \\
    装配生产线 & 0.2/条 & 4     & 0.8   & 阿里巴巴报价 \\
    桌椅    & 0.0378/套 & 10    & 0.378 & 京东商城报价 \\
    电子计算机 & 0.3999/台 & 4     & 1.5996 & 京东商城报价 \\
    压铸模具(开模) & 5/个   & 1     & 5     & 西安模具厂报价 \\
    塑料模具(开模) & 0.12/个 & 1     & 0.12  & 西安模具厂报价 \\
    合计    &       &       & 27.8976 &  \\
    \bottomrule
    \end{tabular}%
  \label{tab:addlabel}%
\end{table}%

\subsubsection{经营计划}
第一年内完成公司的组建,厂房的建设,并在西安市内推广、销售产品.

第三年内完成公司重组,同时稳固西安市的市场,并向全省推广.

第四年内实现公司的规范化,专业化,商业化.完善公司网站的建设,稳定运行,并改建公司场地,业务在全省扎实,评判具有影响力.

第六年开始全面向全国进军.


\subsubsection{资本退出}
风险投资的特点是阶段性投资,资本退出是投资者的首要目标.风险资金 退出的成功与否关键取决于公司的业绩和发展前景.本公司将以非常负责的态 度对待我们的投资者,将把投资者在退出时得到尽可能大的回报放在十分重要的 位置上.鉴于企业项目特色,我们设计如下的风险资金退出方式,旨在平衡企 业发展与福利分享,力保风险投资机构收益最大化与本企业的长久发展,实现 双方的互利双赢.

\begin{enumerate}
\item 股权转让

      预计我公司运营至第12个月时可以收回成本,从这时起,风险投资可以逐步转 让其持有股份,达到原始投资保值增值目的.

\item 公司回购股权

      公司运作进入成熟期后,若风险投资商愿意出售股权,公司可以根据公司当时 盈利情况,将高于原股权现价一定百分比进行回购,达到风险资金退出的目的.

\item 国内主板市场上市

      由于我国《公司法》规定, 上市公司的股本总额不得少于5000 万元,且必须达到连续三年盈利的要求.我公司决定在创立5年内暂不考虑上市事宜.

      公司属于有发展前景和增长潜力的企业.当企业达到一定规模,业务面覆盖全省的时候,,可以考虑在主板上市,风险资金则可以通过公开上市退出.

\end{enumerate}

综合以上分析,公司经过了创建期、发展期,已完成相关产品的开发和市场的 占领,同时,公司在全省乃至全国油烟机领域树立了良好的形象,将有相当的知 名度.预计风险资金于第五年左右退出本公司较合适.无论采用何种退出机制,我们都将兼顾风险投资商和企业各方的利益.

\subsubsection{投资项目价值评估}
对投资项目价值进行评估是投资前的重要准备工作,为了降低投资风险,获得最佳投资效益,需要对投资项目做出正确的评价.
对项目的投资效果进行经济评价的方法,有静态分析法和动态分析法.静态分析法在对投资项目作经济分析时,不考虑资金的时间价值,主要有投资回收期法、投资收益率法等.动态分析法也叫贴现法,主要包括净现值法、内部 收益率法、获利能力指数法等方法.因为它考虑了资金的时间价值,较静态分 析法更为实际,合理.
我们将结合静态与动态分析法,对公司的此项投资做一评估.

\begin{enumerate}
\item 折现率的确定

      折现率是运用收益现值法评估企业价值时的重要参数之一,确定折现率是 进行动态分析的前提,折现率主要有累加法、市场比较法、社会平均收益率法 等确定方法. 综合考虑各种方法的特点,我们决定采用累加法估计折现率.累加法是折 现率确定中最常用的方法.一般来说,折现率应包含无风险报酬率、风险报酬 率和通货膨胀率.无风险报酬率是指资产在一般条件下的获利水平;风险报酬 率是指冒风险取得的报酬与资产的比率.对于非亏损行业,整体企业的未来收 益额可根据未来税后净利润或净现金流量来预测,折现率可以行业平均资金利 润率为基础,再加上3%~5%的风险率来确定.除有确凿证据表明具有高风险外, 折现率一般不高于15\%. 参考市场无风险利率大概在5\%左右,我们决定将折现率定为9.7\%.

\item 现金流及折现现金流

      按照9.7\%的折现率计算我公司未来五年内的现金流及折现现金表如下所示:

      \begin{table}[htbp]
       \centering
       \caption{累计折现净现金流量}
       \begin{tabular}{|lrrrrr|}
       \toprule
        年份   & 1     & 2     & 3     & 4     & 5 \\
        \hline
        净现金流量 & 87.5024 & -32.38 & 16.17 & 719.675 & 2184.481 \\
       折现净现金流量 & 79.76518 & -32.38 & 14.7402 & 656.0392 & 1991.323 \\
       累计折现净现金流量 & 79.76518 & 47.38518 & 62.12538 & 718.1646 & 2709.488 \\
       \bottomrule
      \end{tabular}%
       \label{tab:addlabel}%
       \end{table}%

\end{enumerate}

\subsubsection{财务分析及计划}
\begin{enumerate}
\item 财务预算
      \begin{enumerate}[(1)]
      \item 基本财务假设
            \begin{enumerate}[$\bullet$]
            \item 固定资产折旧

                  固定资产的折旧选择直线折旧法,即将固定资产原值扣除净残值后在预计 使用年限内平均摊销.预计固定资产使用年限为5年,残值率为5\%,并且固定资 产的折旧计入管理费用.

                  \begin{table}[H]
                  \centering
                  \caption{固定资产折旧预算}
                  \begin{tabular}{|lrrrr|}
                  \toprule
                  资产项目  & 原值(单位:元) & 残值率 & 使用年限 & 年折旧额(单位:元) \\
                  \hline
                  电子计算机 & 15996 & 10\%  & 5     & 1599.6 \\
                  装配生产线 & 8000  & 10\%  & 5     & 800 \\
                  桌椅    & 3780  & 5\%   & 7     & 189 \\
                  共计    &       &       &       & 2588.6 \\
                  \bottomrule
                  \end{tabular}%
                  \label{tab:addlabel}%
                  \end{table}%

            \item 无形资产

                  根据会计准则的规定“无法预见无形资产为企业带来经济利益期限的,应 当视为使用寿命不确定的无形资产,在持有期间内不需要摊销,但需要至少于 每一会计期末进行减值测试”,我公司的技术作为无形资产不进行摊销,但是每年末进行减值测试.

            \item 利润分配

                  股利政策公司前处于初创期时,需要大量资金维持企业运营和发展,因此第一年不分配股利.从第三年开始,公司预计按年末可分配利润的30\% 分配现金股利,第四年开始50\%分配现金股利.这样既保证了公司有足够的资金进行运作、扩大规模,满足长期发展,又考虑了股东的获利需求.

            \item 应收账款

                  假定应收账款为每一年收入的20\%,应收的部分在下一年度收回.

            \item 应缴税款

                  税金根据《中华人民共和国企业所得税法》企业所得税的税率为25\%,营业税为 营业额的5\%,城市维护建设费为营业税的7\% 教育费附加为营业税的3\%.

            \item 法定盈余公积

                  西安TripleHIT有限公司按照公司法的规定,本公司应当按公司净利润(弥补以前年度亏损以后)的10\%提取法定盈余公积.
            \end{enumerate}

      \item 财务预算的预测思路

            适应公司管理需要,建立、健全内部约束机制,规范财务预算管理行为,提高经济效益和财务管理水平,西安TripleHIT有限公司的财务预算运用的是全面预算方法,以收入预测为起点,进而对成本及费用等各个方面进行预测,并在这些预测的基础上,编制出预计的资产负债表、利润表、现金流量表以反映企业在未来五年间的财务状况、经营成果和现金流量状况,并为后面财务指标分析做数据支持.

      \item 营业额预算

            \begin{enumerate}[$\bullet$]
            \item 价格确定

                  我公司产品理念是高水准的用户体验及效果,中段产品的价格.在经过市场大范围调研后,我们将本产品初步定价为2400 元.以后会随着市场的接受以及行业情况调整价格.

            \item 营业收入预测

                  \begin{table}[H]
                  \centering
                  \caption{未来五年营业收入估算}
                  \begin{tabular}{|rrr|}
                  \toprule
                  年份 & 销售量(单位:台) & 销售额(单位:万元) \\
                  \hline
                  1     & 500   & 120 \\
                  2     & 1400  & 336 \\
                  3     & 3000  & 720 \\
                  4     & 11000 & 2640 \\
                  5     & 30000 & 7200 \\
                  \bottomrule
                  \end{tabular}%
                  \label{tab:addlabel}%
                  \end{table}%
            \end{enumerate}

      \item 成本费用预测

            \begin{enumerate}[$\bullet$]
            \item 营业成本及费用预测

                  {\footnotesize{\begin{longtable}{|lrrrrrrrrr|}
                  \caption{未来五年成本总合算} \\
                  \hline
                  年份  & 广告成本  & 运输成本  & 生产成本  & 销售成本  & 售后成本  & 管理成本  & 财务成本  & 存储成本  & 共计 \\
                  \hline
                  1     & 0     & 2     & 22.5  & 58    & 25.6  & 19.5  & 10.4  & 10    & 148 \\
                  2     & 20    & 5.6   & 57.4  & 119.4 & 52.5  & 19.8  & 10.4  & 25    & 310.1 \\
                  3     & 55    & 18    & 114   & 137   & 65.5  & 35    & 22.88 & 55    & 502.38 \\
                  4     & 167   & 77    & 385   & 238   & 143   & 115   & 35.1  & 115   & 1275.1 \\
                  5     & 800   & 210   & 990   & 447   & 357.525 & 120   & 52    & 287.5 & 3264.025 \\
                  \hline
                  \end{longtable}}}

            \item 营业税及附加预测

                  \begin{table}[htbp]
                  \centering
                  \caption{营业税金及附加}
                  \begin{tabular}{|lrrrrr|}
                  \toprule
                  时间    & 1     & 2     & 3     & 4     & 5 \\
                  营业额   & 120   & 336   & 720   & 2640  & 7200 \\
                  营业税   & 6     & 16.8  & 36    & 132   & 360 \\
                  城建税   & 0.42  & 1.176 & 2.52  & 9.24  & 25.2 \\
                  教育税附加 & 0.18  & 0.504 & 1.08  & 3.96  & 10.8 \\
                  合计    & 6.6   & 18.48 & 39.6  & 145.2 & 396 \\
                  \bottomrule
                  \end{tabular}%
                  \label{tab:addlabel}%
                  \end{table}%

            \end{enumerate}
      \end{enumerate}
\end{enumerate}

\newpage
\subsubsection{财务报表}
\begin{enumerate}
\item 损益预估表

      \begin{table}[H]
      \centering
      \caption{利润估算表}
      \begin{tabular}{|llrrrrr|}
      \toprule
      \multicolumn{7}{c}{编制单位:西安TripleHIT吸油烟机有限公司                             单位:万元} \\
      \hline
      \multirow{2}[0]{*}{序号} & \multirow{2}[0]{*}{项目} & \multicolumn{5}{c}{计算期} \\
           &       & 1     & 2     & 3     & 4     & 5 \\
      \hline
      1     & 营业收入  & 120   & 336   & 720   & 2640  & 7200 \\
      2     & 营业税金及附加 & 6.6   & 18.48 & 39.6  & 145.2 & 396 \\
      3     & 总成本费用 & 148   & 310.1 & 502.38 & 1275.1 & 3264.025 \\
      4     & 利润总额(1-2-3) & -34.6 & 7.42  & 178.02 & 1219.7 & 3539.975 \\
      5     & 弥补以前年度亏损 & 0     & 7.42  & 27.18 & 0     & 0 \\
      6     & 应纳税所得额(5-6) & -34.6 & 0     & 105.4 & 1219.7 & 3539.975 \\
      7     & 所得税   & 0     & 0     & 26.35 & 304.925 & 884.9938 \\
      8     & 净利润   & -34.6 & 0     & 79.05 & 914.775 & 2654.981 \\
      9     & 期初未分配利润 & 0     & 0     & 0     & 49.8015 & 434.0594 \\
      10    & 可供分配利润 & -34.6 & 0     & 79.05 & 964.5765 & 3089.041 \\
      11    & 提取法定盈余公积金 & 0     & 0     & 7.905 & 96.45765 & 308.9041 \\
      12    & 可供投资者分配的利润 & -34.6 & 0     & 71.145 & 868.1189 & 2780.137 \\
      13    & 各投资方分配利润 & 0     & 0     & 21.3435 & 434.0594 & 1390.068 \\
            & 甲方    & 0     & 0     & 14.229 & 289.373 & 926.7122 \\
            & 乙方    & 0     & 0     & 7.1145 & 144.6865 & 463.3561 \\
      14    & 未分配利润 & 0     & 0     & 49.8015 & 434.0594 & 1390.068 \\
      15    & 息税前利润 & -28   & 25.9  & 217.62 & 1364.9 & 3935.975 \\
      \bottomrule
      \end{tabular}%
      \label{tab:addlabel}%
      \end{table}

\newpage
\item 现金流预测

      \begin{table}[H]
      \centering
      \caption{计划现金流量表}
      \begin{tabular}{|rlrrrrr|}
      \toprule
      \multicolumn{7}{c}{编制单位:西安TripleHIT吸油烟机有限公司                             单位:万元} \\
      \hline
      \multicolumn{1}{c}{\multirow{2}[0]{*}{序号}} & \multicolumn{1}{c}{\multirow{2}[0]{*}{项目}} & \multicolumn{5}{c}{计算期} \\
            &       & 1     & 2     & 3     & 4     & 5 \\
      \hline
      1     & .经营活动净现金流量 & -34.6 & 7.42  & 151.67 & 914.775 & 2654.981 \\
      1.1   & 营业收入  & 120   & 336   & 720   & 2640  & 7200 \\
      1.2   & 现金流出  & 154.6 & 328.58 & 568.33 & 1725.225 & 4545.019 \\
      \multicolumn{1}{l}{1.2.1} & 经营成本  & 148   & 310.1 & 502.38 & 1275.1 & 3264.025 \\
      \multicolumn{1}{l}{1.2.2} & 营业税金及附加 & 6.6   & 18.48 & 39.6  & 145.2 & 396 \\
      \multicolumn{1}{l}{1.2.3} & 所得税   & 0     & 0     & 26.35 & 304.925 & 884.9938 \\
      2     & 投资活动净现金流量(2.1-2.2) & -77.8976 & -89.8 & -135.5 & -195.1 & -470.5 \\
      2.1   & 现金流入  & 0     & 0     & 0     & 0     & 0 \\
      2.2   & 现金流出  & 77.8976 & 89.8  & 135.5 & 195.1 & 470.5 \\
      \multicolumn{1}{l}{2.2.1} & 建设投资  & 7.8976 & 4.8   & 10.5  & 5.1   & 20.5 \\
      \multicolumn{1}{l}{2.2.2} & 维护运营投资 & 20    & 20    & 50    & 70    & 100 \\
      \multicolumn{1}{l}{2.2.3} & 流动资金  & 50    & 65    & 75    & 120   & 350 \\
      \multicolumn{1}{l}{2.2.4} & 其他流出  & 0     & 0     & 0     & 0     & 0 \\
      3     & 筹资活动净现金流量(3.1-3.2) & 200   & 37.625 & -33.2435 & -445.484 & -1411.02 \\
      3.1   & 现金流入  & 200   & 50    & 0     & 0     & 0 \\
      \multicolumn{1}{l}{3.1.1} & 项目资本金投入 & 100   & 0     & 0     & 0     & 0 \\
      \multicolumn{1}{l}{3.1.2} & 长期借款  & 0     & 50    & 0     & 0     & 0 \\
      \multicolumn{1}{l}{3.1.3} & 风险投资  & 100   & 0     & 0     & 0     & 0 \\
      3.2   & 现金流出  & 0     & 12.375 & 33.2435 & 445.4844 & 1411.018 \\
      \multicolumn{1}{l}{3.2.1} & 各种利息支出 & 0     & 2.375 & 1.9   & 1.425 & 0.95 \\
      \multicolumn{1}{l}{3.2.2} & 偿还债务本金 & 0     & 10    & 10    & 10    & 20 \\
      \multicolumn{1}{l}{3.2.3} & 应付利润(股利分配) & 0     & 0     & 21.3435 & 434.0594 & 1390.068 \\
      4     & 净现金流量 & 87.5024 & -32.38 & 16.17 & 719.675 & 2184.481 \\
      5     & 累计盈余资金 & 87.5024 & 55.1224 & 71.2924 & 790.9674 & 2975.449 \\
      \bottomrule
      \end{tabular}%
      \label{tab:addlabel}%
      \end{table}%

\newpage
\item 资产负债估算

      \begin{table}[H]
      \centering
      \caption{资产负债预估表}
      \begin{tabular}{|llrrrrr|}
      \toprule
      \multicolumn{7}{c}{编制单位:西安TripleHIT吸油烟机有限公司                             单位:万元} \\
      \hline
      \multicolumn{1}{c}{\multirow{2}[0]{*}{序号}} & \multicolumn{1}{c}{\multirow{2}[0]{*}{项目}} & \multicolumn{5}{c}{计算期} \\
          &       & \multicolumn{1}{c}{1} & \multicolumn{1}{c}{2} & \multicolumn{1}{c}{3} & \multicolumn{1}{c}{4} & \multicolumn{1}{c}{5} \\
      \hline
      \multicolumn{1}{r}{1} & 资产总额  & 265.6958 & 203.2362 & 319.8672 & 1108.623 & 3572.585 \\
      \multicolumn{1}{r}{1.1} & 流动资产总额 & 157.7613 & 90.27977 & 196.6361 & 981.9279 & 3427.047 \\
      1.1.1 & 货币资金  & 138.5613 & 36.51977 & 81.43612 & 559.5279 & 2275.047 \\
      1.1.2 & 应收账款  & 19.2  & 53.76 & 115.2 & 422.4 & 1152 \\
      \multicolumn{1}{r}{1.2} & 固定资产  & 7.63874 & 12.28137 & 22.43765 & 26.57717 & 45.47857 \\
      1.2.1 & 固定资产原值 & 7.8976 & 12.6976 & 23.1976 & 28.2976 & 48.7976 \\
      1.2.2 & 减;累计折旧 & 0.25886 & 0.416228 & 0.75995 & 1.720428 & 3.319028 \\
      1.2.3 & 固定资产净值 & 7.63874 & 12.28137 & 22.43765 & 26.57717 & 45.47857 \\
      \multicolumn{1}{r}{1.3} & 无形资产及其他资产 & 100.2958 & 100.6751 & 100.7934 & 100.1183 & 100.0592 \\
      1.3.1 & 无形资产原值 & 100   & 100   & 100   & 100   & 100 \\
      1.3.2 & 长期待摊费用 & 0.2958 & 0.675088 & 0.793408 & 0.11832 & 0.05916 \\
      \multicolumn{1}{r}{2} & 负债及所有人权益 & 265.6958 & 203.2362 & 319.8672 & 1108.623 & 3572.585 \\
      \multicolumn{1}{r}{2.1} & 流动负债总额 & 0     & 2.375 & 1.9   & 1.425 & 0.95 \\
      2.1.1 & 应付利息  & 0     & 2.375 & 1.9   & 1.425 & 0.95 \\
      \multicolumn{1}{r}{2.2} & 长期负债总额 & 0     & 40    & 30    & 20    & 0 \\
      \multicolumn{1}{r}{2.3} & 所有人权益 & 265.6958 & 158.4862 & 286.0672 & 1085.773 & 3570.685 \\
      2.3.1 & 资本金   & 300   & 300   & 300   & 300   & 300 \\
      2.3.2 & 累计盈余公积金 & 0     & 0     & 7.905 & 96.45765 & 405.3617 \\
      2.3.3 & 累计未分配利润 & 0     & 0     & 49.8015 & 483.8609 & 1873.929 \\
      \bottomrule
      \end{tabular}%
      \label{tab:addlabel}%
      \end{table}%
\end{enumerate}

\subsubsection{财务指标分析}
\begin{enumerate}
\item 盈利能力分析

      \begin{enumerate}[(1)]
      \item 盈利能力分析

            根据ROI公式计算可得我公司ROI=315.2\%,这个数字远高于行业水平,说明我公司良好盈利水平.

      \item 投资回收期

            根据投资回收期的计算公式,我们可计算出本公司投资回收期Pt=3.24年,即公司在第四年可收回全部投资,这个时间可以令投资者接受.
      \end{enumerate}

\item 售利润率

      自公司实现盈利起,公司的销售利润率计算如下:

      \begin{table}[H]
      \centering
      \caption{销售利润表}
      \begin{tabular}{|lrrrrr|}
      \toprule
      年份    & 1     & 2     & 3     & 4     & 5 \\
      \hline
      利润总额  & -34.6 & 0     & 79.05 & 914.775 & 2654.981 \\
      营业收入  & 120   & 336   & 720   & 2640  & 7200 \\
      销售利润率 & -28.83\% & 0.00\% & 10.98\% & 34.65\% & 36.87\% \\
      \bottomrule
      \end{tabular}%
      \label{tab:addlabel}%
      \end{table}%

      从表中可以看出,除了营业第一、二年以外,公司利润率不断上升,切维持在较高水平.第二年至第四年是公司大规模拓展业务阶段,作为新兴的油烟机公司,我们的产品被市场接受会有一定时间,但只要有了知名度,销售利润率增长就会更快.总的来说,公司的销售利润 率较高,前景乐观.

\item 偿债能力分析

      公司五年的资产负债率如下:

      \begin{table}[H]
      \centering
      \caption{资产负债表}
      \begin{tabular}{|lrrrrr|}
      \toprule
      年份    & 1     & 2     & 3     & 4     & 5 \\
      \hline
      负债合计(万元) & 0     & 40    & 30    & 20    & 0 \\
      资产合计(万元) & 265.6958 & 203.2362 & 319.8672 & 1108.623 & 3572.585 \\
      资产负债率 & 0.00\% & 19.68\% & 9.38\% & 1.80\% & 0.00\% \\
      \bottomrule
      \end{tabular}%
      \label{tab:addlabel}%
      \end{table}%

      公司的资产负债率总体维持在较低水平,且呈逐年下降 趋势,反映了债权人发放贷款的安全可以得到保障,也说明公司有能力合理利用债权人资金进行经营活动.
\end{enumerate}