\section{产品与服务}
本节将对核心产品——基于MicroPython的PYBoard开发板进行详细的原理说明.
\subsection{研究开发}
\subsubsection{产品描述}
MicroPython是包括的Python标准库,并可在微控制器和约束环境下实现优化运行的精干高效的Python的编程语言.
MicroPython兼容性优良,可以让用户将代码从电脑易于移植进单片机或嵌入系统.

\begin{figure}[H]
\centering
\includegraphics[width=12cm]{3.0.jpg}
\caption{MicroPython}
\label{MicroPython}
\end{figure}

用户通过MicroPython可以轻松实现对微控制器的控制,不需要通过复杂的编程,可直接通过MicroPython脚本语言进行操作.
不同于传统的微控制器控制需要重新修改、编译、上传程序,MicroPython脚本语言可实现实时的操作,去掉繁琐的流程,操作更加简单化.
用户完全可以通过MicroPython语言实现硬件底层的访问和控制,比如说控制LED 灯泡、LCD 显示器、读取电压、控制电机、访问SD卡等.
由于MicroPython的简约特性,用户完全可以在学习脚本语言的同时,进行个人兴趣DIY制作,减少了学习时间和成本,增强了编程的体验感,在最短时间内看到学习成果的体现.

以Otto机器人为例,借助MicroPython平台,用户可以自主设计小型玩具机器人,简单且易于操作,互动性和趣味性较强。

\begin{figure}[H]
\centering
\subfigure[]{
\begin{minipage}{7.5cm}
\centering
\includegraphics[width=7.5cm]{3_otto01.jpg}
\hspace{1cm}                        
\end{minipage}%
}%
\subfigure[]{
\begin{minipage}{7.5cm}
\centering
\includegraphics[width=7.5cm]{3_otto02.jpg}
\hspace{1cm}
\end{minipage}%
}%
\caption{基于MicroPython的Otto机器人}
\label{基于MicroPython的Otto机器人}
\end{figure}

\subsubsection{产品原理}
\begin{enumerate}[(1)]
\item MicroPython编译原理

      单片机的编程环境默认为C语言。目前在单片机业内有两种编译机制,主流是以Arduino为代表的AVR16单片机编译机制:

     \begin{figure}[H]
     \centering
     \includegraphics[width=12cm]{3_Cbin.pdf}
     \caption{基于C++的单片机编译机制}
     \label{基于C++的单片机编译机制}
     \end{figure}

     而借助python语言编译过程中虚拟机(VM-Visual Machine)的特性,可以使编译速度大大加快。python更偏向解释型语言,虚拟机即为Python 的解释器,模拟可执行程序X86机器上的运行,像CPU一样将字节码一条一条的执行。

      \begin{figure}[H]
      \centering
      \includegraphics[width=12cm]{3_Pbin.pdf}
      \caption{基于MicroPython的单片机编译机制}
      \label{基于MicroPython的单片机编译机制}
      \end{figure}

      C语言对于普通用户来说,学习成本是较大的。而Python语言需要在PC或者其他嵌入式linux平台才可以运行,在单片机,比如说STM32,是不可运行的。所以MicroPython的出现,综合了单片机使用C语言的局限和Python的简单性,帮助用户解决了这个问题。MicroPython为面向对象的语言,将用户界面输入的python语句进行封装,利用VM虚拟机实时转换为单片机可识别的*.bin文件,使得python语言可在例如STM32之类的单片机上运行。
      
\item PYboard运行原理

      PYboard是MicroPython官方单片机支持电路板,编程环境为MicroPython。PYboard基于STM32F405RG微控制器,通过板载的microUSB接口供电以及进行数据传输,板载外设包括了一个MicroSD卡座接口、4个LED灯、两个机械按键、一个加速传感器、时钟模块,其它外设信号都通过板子上的金属通孔引出,具体的信号定义可以参考下图。
      
      \begin{figure}[H]
      \centering
      \includegraphics[width=12cm]{3_pyboard.jpg}
      \caption{PYboard信号输出}
      \label{PYboard信号输出}
      \end{figure}
      
      以STM32单片机为例,通过MicroUSB线连接上电脑后,设备管理器出现了一个需要安装驱动的虚拟串口,从电脑中看到pyboard的U盘。U盘中的有几个重要的文件boot.py、main.py、pybcdc.inf。若板子正常启动,上电先会运行boot.py,然后再配置USB,最后运行main.py。其中pybcdc是需要安装的驱动。
      
      \begin{figure}[H]
      \centering
      \includegraphics[width=12cm]{3_upan.jpg}
      \caption{U盘中的文件}
      \label{U盘中的文件}
      \end{figure}
      
      此时,将{\bf{MicroSD卡}}插入上电,pyboard会默认{\bf{从SD卡启动}}来代替原本的微控制器中的ROM中启动。
      
      \begin{figure}[H]
      \centering
      \includegraphics[width=12cm]{3_sd.jpg}
      \caption{最终连接状态}
      \label{最终连接状态}
      \end{figure}
      
      在SD卡中,通过使用python语言编辑main.py文件,添加相应脚本代码,就可以实现上电即运行程序的操作。
\end{enumerate}


\subsubsection{未来产品与规划}
\begin{enumerate}[(1)]

\item 初期产品

      初期团队规模较小、技术水平不够高、资金资源有限,在满足用户核心需求的基础上,将进行主要的功能实现.同时,为践行“创新”与“颠覆”的产品理念,初期产品的主要任务在于带来全新的用户使用体验,从而建立品牌效应,为中后期产品的推出打好基础.在单片机市场逐渐被Arduino平台占领,且同期无其他替代产品出现时,MicroPython的出现是极具颠覆性的.

      基于以上考虑,初期产品主要偏向于采用MicroPython平台的开发,建立自身的python库,为用户建立强大的代码支持,提升自身信誉度.

\item 中期产品

      在得到稳定市场份额、确定目标人群、积累品牌效应的基础上,中期产品将更多地体现智能化设计,在中端产品中争取更大的话语权.中期产品将对初期产品进行优化升级,在硬件方面针对用户反馈与售后信息进行修正改动,在软件方面进行功能拓展与性能提升.在保证系列产品较小的价位浮动的前提下,尽量提升用户体验,扩大目标人群数量.

      基于以上考虑,中期产品可以考虑加入单片机生态链以及创客教育,与国内几大创客品牌教育达成合作,一方面获取更多的用户需求的资料后,可以对不同需求,例如信息安全方向的wifi欺骗器,及时扩充python库;一方面利用自身优势扩大市场,与arduino平台达成良性竞争。

\item 终期产品

      在经过初期和中期产品的多层迭代后,MicroPython平台已日趋完善。此时可以考虑利用周边效应,建立自身创客教育品牌——极客教育.将前期所有研发成果转变成线上线下课堂的形式,加入新媒体模式,与MicroPython平台相辅相成,融入国内STEM生态圈。

\end{enumerate}

\subsection{资金需求}



\subsection{服务支持}

\begin{enumerate}[(1)]
\item PYboard安装

      随产品会配发相应说明书,网站上放置链接以供用户参考。

\item Python脚本语言编写

      MicroPython平台提供基本案例和教程。用户还可以在GitHub平台上与其他用户交流,查看其他用户编写的代码。
      \url{https://github.com/micropython}

\item 官方售后中心

      设立专门的组织结构,选派业务素质高、责任心强的人员负责售后管理,实施定人、定责、定任务、定范围的岗位责任制.讨论建立定期巡回制度、质量跟踪制度、定期座谈会制度、来访接待制度、零配件供应制度、“包退、包换、包修”制度、销售档案制度等,明确服务职责,规范服务行为,完善服务流程.
\end{enumerate}
