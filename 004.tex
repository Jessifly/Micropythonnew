\section{行业与市场}
\subsection{行业标准}
\begin{enumerate}
\item 2011年10月 国家标准化委员会颁布国家标准GB/T 17713-2011《吸油烟机》,,它规定了吸油烟机的术语和定义、产品分类、要求、试验方法、检验规则及标志、包装、运输、贮存的要求.相对于旧版,新增气味降低度和油脂分离度这两大指标的数据(外排式油烟机常态气味降 低度应≥90\%,且瞬时气味 降低度应≥50\%,油脂分离 度应≥80\%).
\item 2011年8月 住建部颁布 GB50096-2011《住宅 设计规范》,规定定厨房应设置吸油烟机等设施或为其预留位置.使吸油烟机成为住宅整体厨房系统中的重要组成部分,为吸油烟机市场营造了更大的发展空间.
\item 据了解, 白2012年4月9日起,家用吸油烟机须采用新版认证依据实施认证.对于持有有效的家用吸油烟机节能环保认证证书的认证委托人,应于2012年4月9日起,向中国质量认证中心提交转换新版技术规范节能环保认证证书的变更申请,并接受依据新版技术规范对实验样品进行检测,旧版技术规范认证证书转换工作应于2013 年4 月10 日前完成,逾期未完成转换的认证证书将予以暂停,2013 年7 月10日前仍未完成换版的证书,将予以撤销.对于换证期内发生的变更,原则上采用新版技术规范.
\end{enumerate}

{\footnotesize{\begin{longtable}{l|l|l}
\caption{国外吸油烟机开发技术标准}  \\
\hline
分类 & 标准号 & 名称 \\
\hline
\multirow{2}{*}{安全要求} & IEC 60335-1-2010 & 家用和类似用途电器的安全 第1部分 \\
\cline{2-3}
 & IEC 60335-2-31-2009 & 家用和类似用途电器的安全 第2-31 部分  \\
 \hline
空气动力要求  & IEC 61591:1997+a1:2005 & 家用吸油烟机性能测试方法  \\
\cline{2-3}
油脂分离度 & IEC 60665-1980 & 家用和类似用途的交流排气扇及其调速器\\
\cline{2-3}
气味降低度 & DIN 44971[1992] & 家用吸油烟机概念、检验和要求\\
 \hline
\multirow{2}{*}{噪声} & IEC 60704-2-13-2011 & 家用和类似用途电器的测定空中传播噪声的试验规范 第2-13部分 \\
\cline{2-3}
 & IEC 60704-1-2010 & 家用和类似用途电器的安全 测定空中传播噪声的试验规范 第1部分  \\
\hline
\end{longtable}}}

{\footnotesize{\begin{longtable}{l|l|l|l}
\caption{国内吸油烟机开发技术标准}  \\
\hline
分类 & 标准号 & 名称 & 引用标准\\
\hline
\multirow{2}{*}{安全要求} & GB 4706.1-2005 & 家用和类似用途电器的安全 第1部分 & IEC 60335-1-2004 \\
\cline{2-4}
 & GB 4706.28-2008 & 家用和类似用途电器的安全 第2-31 部分& IEC 60335-2-31-2006  \\
\hline
空气动力要求  & \multirow{3}{*}{GB/T 17713-2011} & \multirow{3}{*}{家用吸油烟机}  \\
油脂分离度 &   &  \\
气味降低度 &   &  \\
\hline
噪声 & GB/T 4214.1-2000 & 声学家用电器测试方法 & IEC 60704-1.1997 \\
\hline
\end{longtable}}}

\subsection{市场调研}
中国抽油烟机行业经过多年发展,已成为一个较成熟产业.与其它家电产品相比,这一产业仍处于一个相对激烈的竞争环境里,市场需求较大.2011 年1-12月全国家用吸排抽油烟机产量达1895万台,2012年1-7月达到了1065万台,比2011年同期增长了4.73\%.2013年抽油烟机线上市场继续放量.据统计数据显示,吸油烟机线上零售量和零售额分别出现了150\% 和200\% 的高速增长,分别达到160万台和24亿元.不同于其他家电产品的量额增长同步,{\bf{吸油烟机线上零售额的增幅明显高于零售量}}.随着企业对居民烹饪习惯和厨房环境的深入研究,随着消费者对生活品质提升的追求,抽油烟机在产品种类、技术含量、标准化等方面都取得了一定的成就,我国抽油烟机行业已经逐渐迈入{\bf{品牌与技术主导的新阶段}}.目前,我国抽油烟机产品主要分为四类:{\bf{中式、欧式、侧吸式和近吸式}},品牌方面则以珠江三角洲和长江三角洲为代表的各大抽油烟机生产企业占据了国内抽油烟机市场上的主要份额,老板、方太、万和、华帝、美的、帅康等成为抽油烟机市场主流品牌.

{\footnotesize{\begin{longtable}{|l|l|l||l|l|l|}
\caption{2006-2015年中国家用吸排油烟机产量及增长率统计表}  \\
\hline
    时间    & 年度产量/万台 & 同比增长/\% & 时间    & 年度产量/万台 & 同比增长/\%  \\
    \hline
    2006年 & 1089.87 & 45.74 & 2007年 & 1189.07 & 7.64 \\
    2008年 & 1633.15 & 23.42 & 2009年 & 1684.74 & 2.68 \\
    2010年 & 1927.28 & 10.61 & 2011年 & 1895.06 & 16.65 \\
    2012年 & 2106.7 & 4.89 & 2013年 & 2559.42 & 22.35 \\
    2014年 & 2939.73 & 12.52 & 2015年 & 2887.28 & -1.78 \\
  \hline
\end{longtable}}}

\begin{figure}[H]
\centering
\includegraphics[width=12cm]{4.2.jpg}
\caption{吸油烟机十年销量}
\label{吸油烟机十年销量}
\end{figure}

经过实地市场调查,我们发现抽油烟机市场主要呈现出以下几个特点:

\begin{enumerate}
\item 市场格局相对稳定,品牌竞争激烈

      吸油烟机既是一个准入门槛较低的行业,同时又是一个激烈竞争淘汰率较高的行业,自进入二十一世纪之后,随着市场经济的成熟,消费者对吸油烟机的需求逐步稳定,吸油烟机厂家在经过几轮淘汰之后,也发展成为一个从原材料供应、模具制造、零配件配套、整机组装直到销售的完整产业链,吸油烟机产品本身也分化成为适应不同消费群体的产品群体,一些业内比较知名的吸油烟机品牌业已成形,如老板,方太,帅康,华帝等品牌.而消费者在选择吸油烟机时,更加趋向于理性消费,往往要对品牌差异化、外观、结构、材料、功能、性能指标、性价比进行全面考查后才购买,可以说从整体来看,吸油烟机产业已进入一个相对稳定发展阶段.但在市调过程中我们发现,在这个阶段中各品牌的市场竞争更为激烈.

\item 油烟机市场同质化现象严重

      在此次市调中,我们发现油烟机市场同质化现象严重.吸油烟机市场同质化现象又表现在设计思想、结构设计、外观造型等许多方面.

      在设计思想上,目前在市场上销售的吸油烟机都是采用以涡轮直接抽排油烟的工作方式,即在抽排油烟过程中,油烟必须通过涡轮的加速作用后才能排出,因此采用这种工作方式的吸油烟机无一例外地会在其涡轮叶面、蜗壳内表面等处形成凝油;

      \begin{figure}[H]
      \centering
      \includegraphics[width=8cm]{4.3.jpg}
      \caption{涡轮直接抽排油烟结构}
      \label{涡轮直接抽排油烟结构}
      \end{figure}

      在产品结构上,虽然我们在商场中可以看到各种品牌不同机型的许多产品,但这些油烟机仔细一看都差不多(西门子、方太、帅康除外).几乎所有的吸油烟机品牌都宣布说自己产品是“特氟龙不沾油涂料喷涂”,所以“涡轮可以方便拆洗”,或者是采用的油烟过滤手段,“永远不需清洗涡轮”,“大风量、高静音”、“欧式风格,现代时尚造型”等等,而不同品牌相同功能和款式的产品价格却相差很大.

      在外观造型上,只有大品牌自成一体,众多小品牌相对趋同;在机型上主要还是以中式机和欧式机为主,虽然也有个别品牌创新突破,但数量不多,只停留在概念机的层面上.值得一提的是老板最新推出的一款新型概念机:尊朗9508,可以视听的功能试图重新定义厨房的概念,其28000元的售价也给市场带来诸多的猜想.

      \begin{figure}[H]
      \centering
      \subfigure[中式机]{
      \begin{minipage}{6.0cm}
      \centering
      \includegraphics[width=6.0cm]{4.41.jpg}
      \hspace{1cm}
      \end{minipage}%
      }%
      \subfigure[欧式机]{
      \begin{minipage}{7.5cm}
      \centering
      \includegraphics[width=7.5cm]{4.42.jpg}
      \hspace{1cm}
      \end{minipage}%
      }%
      \caption{典型机型}
      \label{典型机型}
      \end{figure}

\item 近吸式与大风量受宠

      分产品类型来看,近吸式产品快速崛起.独特的造型和对空气动力学的独特把握使近吸式产品与消费者的需求契合度更高,其零售量的同比增速虽不及欧式产品,但零售量和零售额份额均已突破50\%,将平顶式产品、欧式产品和深罩式产品远远甩在身后.不仅如此,近吸式产品的市场占比还在继续扩大,相比上一年,其零售量份额上升了近14个百分点.近吸式产品的强势增长不仅改变了行业的产品结构,对企业间的品牌竞争也产生了重要影响.市场份额方面紧随近吸式产品的是欧式产品,但其2013年的市场份额下滑了3\%.深罩式和平顶式产品的市场占比也小幅下降,后者的市场份额甚至跌到不足3\%,面临被市场淘汰的风险.

      一般来说,排风量越大,厨房里的油烟吸得越快越干净,所以消费者在购买吸油烟机时会尽可能选排风量大的吸油烟机,这一点在线上线下都得到了充分体现.从排风量来看,$15m^3$至$16.9m^3$、$17m^3$及$17m^3$以上的产品构成了线上的主流,两者累计市场占比超过62\%.相比其他排风量段,这两个细分市场的增速也是最快的,分别达到188\% 和179\%.$15m^3$以下细分市场的份额大幅缩水,累计减少了近18个百分点,并且在未来还将不断受到$15m^3$以上大风量段产品的猛烈冲击.其中$13m^3$至$14.9m^3$风量段产品的市场份额的萎缩最为明显,同比大幅下滑了14个百分点,$13m^3$风量段产品的零售量份额也下降了近4个百分点.

\item 超大型连锁终端垄断一级市场,二级市场遭地方强势终端的顽强阻击

      在一级市场,由于超大型连锁卖场的垄断地位日益稳固,使这类终端成为各品牌实现销售的主要阵地.如:全国连锁的有,苏宁、国美、永乐、五星等;区域连锁的有,武汉工贸,北京大中,长沙通程等.

\item 各品牌开始多路出击,开辟新的渠道

      近一年来,这些大型卖场纷纷进军二级市场,对当地传统卖场造成了很大的冲击.然而,这些实力强大的超级连锁卖场也在一些二级市场出现了水土不服的现象,遭到了当地传统强势卖场的顽强阻击,在有的地方甚至难以生存.如孝感的孝武电器;荆门的东方电器;仙桃的飞达电器;许昌的胖东来;洛阳的容威电器、八方电器;开封的万宝电器等.分析这些传统卖场之所以能够成功的原因,汇总起来有以下几个优势:

      \begin{enumerate}[(1)]
      \item 在当地经营时间比较长,深受当地消费者的信任.
      \item 有地方背景,在各个方面能够得到不同程度的支持和方便.
      \item 由于企业规模相对较小,经营更加灵活多变.
      \item 经销商的销售成本较低,更加有利可图,更愿意与卖场合作.
      \end{enumerate}

      同时,由于超大型零售卖场在市场上的这种垄断,也使经销商和厂家在上述零售终端的利润越来越薄.而且越来越多的品牌在瓜分着有限的卖场资源,使相当一部分品牌在卖场里苦不堪言.现在,许多品牌已经将目光转向其他尚未完全开发的销售通路.各品牌专卖店如雨后春笋在各地陆续建立,其中樱花已经将销售重点由大卖场转向品牌专卖店;小区推广越来越成为各大品牌争夺市场占有率的手段,甚至华帝已经开发出专门用于小区推广的专卖车;多渠道并进已经成为一个成功品牌必须的营销手段.

\item 市场容量与产品投放量同时快速增长

      随着各地房地产业的蓬勃发展,油烟机市场的需求量伴随着商品房的大量销售而增加.这也是近两年油烟机产业快速发展的主要原因之一.这为新品牌进入油烟机市场提供了很多的机会.应该说,油烟机市场依然是一个快速增长的市场,不同层次的品牌在一定时期内还可以共同生存下去.但随着各大品牌对市场投入力度的不断增大,油烟机市场竞争也在逐步增大,并趋于激烈.留给无差异特点的新品牌进入市场的机会已经不多了.

\end{enumerate}

\subsubsection{行业现状}
结合市场调研,可以得到以下结论:

\begin{enumerate}
\item 高端产品占比逐年增加,品牌化格局正在形成

      \begin{figure}[H]
      \centering
      \includegraphics[width=12cm]{4.5.jpg}
      \caption{市场品牌分析}
      \label{市场品牌分析}
      \end{figure}

      消费者对抽油烟机产品的关注主要集中在性能和品牌两个方面,这一特征在高端市场更加明显.就性能而言 ,针对国内家庭烹饪油烟较大的特点,在实现多种性能平衡的基础上,大风量依然是消费者购买吸油烟机产品的首选因素;就品牌而言,近十年来,随着国民经济的发展与生活水平的提高,消费者对抽油烟机的购买需求不断升级,从购买基本“抽烟”功能到购买“品质与美观”,再到追求“健康环保”与“个性化生活方式”,消费行为也从“量”的消费阶段过度到“质”的消费阶段,并逐步上升到品牌消费阶段.

\item 国内品牌处于垄断地位,形成两超多强局面

      一直以来,吸油烟机线下市场的竞争激烈程度相对较低,几大巨头的强势使市场的品牌格局相对稳定.不过随着一些大家电企业对该领域的深度介入,市场已经初露整合调整的迹象,这种状况同样出现在线上.随着大品牌影响力的提升,吸油烟机在线上的品牌集中度还将增强.与线下类似,国内品牌在线上处于绝对垄断地位,零售量排名前10位的品牌中没有一家国外品牌.稍有不同的是,线下的品牌集中度比线上高出不少,这也说明几大吸油烟机巨头在线上的话语权仍低于线下.在吸油烟机新能效标准出台后,一线龙头厂商的线上市场份额或将越做越大,具备一定技术研发能力的二线品牌对主流品牌的挤压还有一定的承受力,但不具备规模优势的小品牌恐怕面临出局的危险.
\end{enumerate}

\subsubsection{目标市场与人群}
\begin{enumerate}
\item 消费者购买因素影响调查:
\begin{enumerate}[(1)]
\item 消费者对权威认证的信任度调查

      \begin{figure}[H]
      \centering
      \includegraphics[width=10cm]{4.61.jpg}
      \caption{消费者对权威认证的信任度调查}
      \label{消费者对权威认证的信任度调查}
      \end{figure}

\item 消费者购买终端选择的调查

      \begin{figure}[H]
      \centering
      \includegraphics[width=9cm]{4.62.jpg}
      \caption{消费者购买终端选择的调查}
      \label{消费者购买终端选择的调查}
      \end{figure}

\item 影响消费者购买因素的调查

      \begin{figure}[H]
      \centering
      \includegraphics[width=12cm]{4.63.jpg}
      \caption{影响消费者购买因素的调查}
      \label{影响消费者购买因素的调查}
      \end{figure}

\item 消费者对油烟机本身要素关注程度的调查

      \begin{figure}[H]
      \centering
      \includegraphics[width=12cm]{4.64.jpg}
      \caption{消费者对油烟机本身要素关注程度的调查}
      \label{消费者对油烟机本身要素关注程度的调查}
      \end{figure}

\item 消费者对油烟机各功能要素关注程度的调查

      \begin{figure}[H]
      \centering
      \includegraphics[width=12cm]{4.65.jpg}
      \caption{消费者对油烟机各功能要素关注程度的调查}
      \label{消费者对油烟机各功能要素关注程度的调查}
      \end{figure}

\item 消费者对油烟机价位选择的调查

      \begin{figure}[H]
      \centering
      \includegraphics[width=12cm]{4.66.jpg}
      \caption{消费者对油烟机价位选择的调查}
      \label{消费者对油烟机价位选择的调查}
      \end{figure}

\item 消费者对现有品牌关注程度的调查

      \begin{figure}[H]
      \centering
      \includegraphics[width=12cm]{4.67.jpg}
      \caption{消费者对现有品牌关注程度的调查}
      \label{消费者对现有品牌关注程度的调查}
      \end{figure}

\item 消费者媒体接触程度的调查

      \begin{figure}[H]
      \centering
      \includegraphics[width=12cm]{4.68.jpg}
      \caption{消费者媒体接触程度的调查}
      \label{消费者媒体接触程度的调查}
      \end{figure}

\item 消费者对新产品命名的调查

      \begin{figure}[H]
      \centering
      \includegraphics[width=12cm]{4.69.jpg}
      \caption{消费者对新产品命名的调查}
      \label{消费者对新产品命名的调查}
      \end{figure}
\end{enumerate}


\item 消费者使用痛点调查分析

      通过华帝股份2015年的调查数据,发现在油烟机使用过程中的痛点中,最大痛点是操作麻烦(82\%),其次是清洗麻烦(63\%),然后是吸力不够(56\%),最后是跑烟问题(41\%).
      油烟机作为功能性特别强的家电产品,基本功能是用户购买吸油烟机时重要的购买决策因素.在问及购买当前使用产品时的考虑因素时,有85\%的受访者表示考虑了较大的排风量,67\%的用户考虑了低噪音,只有33\%的用户考虑了自动清洗.而在使用过程中,受访者反映出的的吸油烟机主要问题是清洁不方便(29\%)、噪音大(22\%)和吸油烟效果不佳(19\%).其中,不易清洗是用户最大的使用痛点,在这29\%的消费者中有七成表示吸油烟机内部如扇叶,涡轮等部位最难清洗.正是由于用户在使用过程中普遍发现难清洗问题,因此在换购吸油烟机,更多的用户把自动清洗功能列为购买决策因素、关注提升了24个百分点达到57\%,尤其是40岁以下的年轻用户对自动清洗功能更为重视.

      \begin{figure}[H]
      \centering
      \includegraphics[width=15cm]{4.6.jpg}
      \caption{抽油烟机购买关注因素}
      \label{抽油烟机购买关注因素}
      \end{figure}

      有网友专门在微博上做了抽油烟机满意度调查,根据此次“吸油烟机满意度大调查”的统计结果显示,有4000多人表示不满意,认为油烟机转速慢,油烟吸不干净,3000 多人表示非常不满意,每次做饭时厨房内油烟肆虐,只有1700多人和1200多人表示满意和非常满意.也就是说,在此次调查中有7成网友不满意自家吸油烟机的表现.

      对于这样的调查结果,业内专家表示,这也是在意料之中.“因为普通油烟机在设计上就存在着诸多缺陷,例如外凸的油网设计就很容易使得油烟反弹,造成二次扩散,从而引起油烟外溢.”上述专家表示,要想解决此类难题,打破常规、颠覆传统设计才是重点.而记者从海尔内部获悉,海尔推出的一款深腔净吸油烟机新品,就彻底打破了常规,颠覆了传统设计,首次采用了深腔式设计,将外凸式的倒梯形油网变成了内凹式的金字塔型油网,增大了集烟腔深度,借鉴漏斗效应,瞬间提高气流上升速度,能彻底解决厨房油烟大的难题,拒绝厨房PM2.5污染.此外,记者还了解到,在网络化消费时代,海尔厨房电器还率先实施了互联网战略,以智慧家庭为中心正在给用户提供网络时代最佳的智慧生活体验.目前不仅建了“一云多网N端”的产业架构,还将每一类产品都变成互联网终端,为用户打造智慧健康的厨房生活空间,让用户无论何时何地都能充分享受智能时代的最佳应用服务.

      \begin{figure}[H]
      \centering
      \includegraphics[width=15cm]{4.7.jpg}
      \caption{抽油烟机满意度调查}
      \label{抽油烟机满意度调查}
      \end{figure}

      中怡康厨卫电器事业部总经理施婷发布《中国厨电用户使用习惯调查研究报告》,施婷表示:从核心用户需求来看,对于吸油烟机被访者主要问题是不容易清洁、噪音大和吸力不够.相对应的用户比较感兴趣的功能是大排风量、静音、自动清洗;另外对不滴漏油、根据油烟大小自动调节运转速度也有较高的关注度.

\end{enumerate}

\subsection{竞争优势}

\subsubsection{同期产品评估}
\begin{enumerate}
\item 外观设计

	   当前市场上主流的抽油烟机大致可以分为中式、欧式和侧吸式.欧式和中式的风扇以及排风系统都位于吸油烟机正下方,正对燃气灶.
       欧式吸油烟机面板较为轻薄,整体体积较小而中式吸油烟机机身体积较大,形似一个大箱子.侧吸式吸油烟机面板倾斜,位于燃气灶斜前方.其他外形设计的吸油烟机市场占有率太低,没有代表性,故在此不予考虑.

      \begin{figure}[H]
      \centering
      \subfigure[中式机]{
      \begin{minipage}{4.5cm}
      \centering
      \includegraphics[width=4.5cm]{4.81.jpg}
      \hspace{1cm}
      \end{minipage}%
      }%
      \subfigure[欧式机]{
      \begin{minipage}{5.5cm}
      \centering
      \includegraphics[width=5.5cm]{4.82.jpg}
      \hspace{1cm}
      \end{minipage}%
      }%
      \subfigure[侧吸式]{
      \begin{minipage}{4.0cm}
      \centering
      \includegraphics[width=4.0cm]{4.83.jpg}
      \hspace{1cm}
      \end{minipage}%
      }%
      \caption{市场主流外观设计}
      \label{市场主流外观设计}
      \end{figure}

\item 性能

	  衡量吸油烟机性能的最重要的指标是排风量和噪音.当前市场上出售的吸油烟机排风量大多在$13-20m^3/s$之间,噪音大小大多在50-60dB之间,
      大致相当于人正常交流时发出的音量大小.

\item 附加功能

	  市场上主流吸油烟机拥有的附加功能一般有自动清洗和远程遥控.自动清洗能允许用户不用亲自动手也不必拆开吸油烟机外壳,吸油烟机就能自动完成内部油污清洗.
      远程操作功能能让用户使用遥控器或者手机app完成对吸油烟机的操作.这两个功能并非所有吸油烟机都有.

\item 价格

	  吸油烟机价格区间较大,从155元到10000元人民币都有.在同等性能下,侧吸式与欧式吸油烟机价格大致相同,且明显高于中式吸油烟机.
      拥有更多附加功能的机型价格比无附加功能的机型明显要高.

\end{enumerate}

\subsubsection{技术创新点}
\begin{enumerate}
\item “两级”式抽油烟

      \begin{figure}[H]
      \centering
      \includegraphics[width=10cm]{4.90.jpg}
      \caption{整体外观创新设计}
      \label{整体外观创新设计}
      \end{figure}
	  两级抽油烟机的外观设计不同于现在市场上出现过的任何吸油烟机.

      一级出风口使用了空气倍增技术,通过空气倍增技术保证大的出风量.
      \begin{figure}[H]
      \centering
      \includegraphics[width=10cm]{4.9.jpg}
      \caption{一级出风设计}
      \label{一级出风设计}
      \end{figure}

      通过对外侧空气的带动作用,在二级截面显著变小的情况下,将油烟顺利抽出,降低抽油烟机纵深比,抛弃了笨重的面板;无叶风扇的使用节约了大量的空间;吸油烟机的整体体积和重量都小于传统吸油烟机,实现轻量化设计的理念.模块化的设计让两级抽油烟机的加工生产、物流运输、维修更换和个人化定制都可以变得十分简便.

      \begin{figure}[H]
      \centering
      \subfigure[二级出风]{
      \begin{minipage}{7.5cm}
      \centering
      \includegraphics[width=7.5cm]{4.101.jpg}
      \hspace{1cm}
      \end{minipage}%
      }%
      \subfigure[气流示意图]{
      \begin{minipage}{7.5cm}
      \centering
      \includegraphics[width=7.5cm]{4.102.jpg}
      \hspace{1cm}
      \end{minipage}%
      }%
      \caption{二级出风设计}
      \label{二级出风设计}
      \end{figure}

\item 排气管出风管道的设计思考

      \begin{figure}[H]
      \centering
      \includegraphics[width=10cm]{4.11.jpg}
      \caption{排气管出风管道}
      \label{排气管出风管道}
      \end{figure}

      使用Comsol和ANSYS软件对模型中的气流场进行仿真,对进气管道和接口处的结构进行了优化,同时通过仿真结果,直观的看到了抽油烟机对外部气流的影响,确保抽油烟机集油烟效果的同时对风力参数进行优化,将每一分风力都用在最需要的地方.

      下图是仿真实验数据图,从上至下显示了优化的过程:

      \begin{figure}[H]
      \centering
      \subfigure[结构1]{
      \begin{minipage}{8.0cm}
      \centering
      \includegraphics[width=8.0cm]{4.121.jpg}
      \hspace{1cm}
      \end{minipage}%
      }%
      \subfigure[风力测试1]{
      \begin{minipage}{6.0cm}
      \centering
      \includegraphics[width=6.0cm]{4.122.jpg}
      \hspace{1cm}
      \end{minipage}%
      }%
      \caption{仿真优化实验一}
      \label{仿真优化实验一}
      \end{figure}

      \begin{figure}[H]
      \centering
      \subfigure[结构2]{
      \begin{minipage}{7.8cm}
      \centering
      \includegraphics[width=7.8cm]{4.131.jpg}
      \hspace{1cm}
      \end{minipage}%
      }%
      \subfigure[风力测试2]{
      \begin{minipage}{6.0cm}
      \centering
      \includegraphics[width=6.0cm]{4.132.jpg}
      \hspace{1cm}
      \end{minipage}%
      }%
      \caption{仿真优化实验二}
      \label{仿真优化实验二}
      \end{figure}

      \begin{figure}[H]
      \centering
      \subfigure[结构3]{
      \begin{minipage}{6.0cm}
      \centering
      \includegraphics[width=6.0cm]{4.141.jpg}
      \hspace{1cm}
      \end{minipage}%
      }%
      \subfigure[风力测试3]{
      \begin{minipage}{6.5cm}
      \centering
      \includegraphics[width=6.5cm]{4.142.jpg}
      \hspace{1cm}
      \end{minipage}%
      }%
      \caption{仿真优化实验三}
      \label{仿真优化实验三}
      \end{figure}
\end{enumerate} 